\documentclass[
      12pt,
      a4paper,
      cleardoublepage=empty,
      final,
      twoside
        ]{scrbook}
%Gummi|065|=)



\usepackage[top=3cm,
      bottom=2cm,
      inner=2.5cm,
      outer=2.5cm,
      bindingoffset=1.5cm
      ]{geometry}

\usepackage{cite}
\usepackage{hyperref}


\usepackage[T1]{fontenc}
\usepackage[utf8]{inputenc}
\usepackage{graphicx}

\usepackage{setspace}
\onehalfspacing

\usepackage{fancyhdr}
\pagestyle{fancyplain}
\fancyhf{}
\lhead[\thesubsection]{\thepage}
\rhead[\thepage]{\leftmark}
\renewcommand{\headrulewidth}{1pt}


\usepackage[explicit]{titlesec}
\titleformat{\chapter}%
            {\fontsize{14}{14}\bfseries}
            {\llap{% label
               \thechapter
               \hskip 9pt}#1}%
            {0pt}% horizontal sep
            {}% before
\titleformat{\section}%
            {\fontsize{12}{12}\bfseries}
            {\llap{% label
               \thesection\hskip 9pt}#1}%
            {0pt}% horizontal sep
            {}% before
\titleformat{\subsection}%
            {\fontsize{12}{12}}% format
            {\llap{% label
               \thesubsection\hskip 9pt}#1}%
            {0pt}% horizontal sep
            {}% before



\renewcommand{\familydefault}{\sfdefault}
\usepackage{helvet}


\title{\fontsize{14}{14}Bachelor Thesis \\ in \\Allgemeine Informatik
\fontsize{22}{22}\textbf{Transhumanism}\\
\fontsize{18}{18} The New Human}
\author{Marcel M\"uller}
\date{\today}

\begin{document}

\setcounter{secnumdepth}{3}


\begin{titlepage}
\begin{center}


\fontsize{18}{18}
\textsc{Wissenschaftliche Arbeit \\[0.5cm] in \\[0.5cm] Allgemeine Informatik}\\[1.5cm]

% Title
\fontsize{22}{22}
{ \bfseries Transhumanism \\[0.4cm] }
\fontsize{18}{18}
{ The new Human\\[10cm] }

% Author and supervisor
\begin{tabular}{l l}
   Referent:  & Hr. Burzlaff \\
   Korreferent: & N/A\\
   Vorgelegt am: & \today\\ % TODO: Put date
   Vorgelegt von: & Marcel M\"uller\\
   & [Matrikelnummer] \\
   & [Strasse und Hausnummer] \\
   & [PLZ und Stadt] \\
   & [HFU-Emailadresse]\\
\end{tabular}

\vfill

% Bottom of the page
{\large \today}

\end{center}
\end{titlepage}

\frontmatter

\section{Preamble}

\begin{par}
  
  I would like to thank the various people who helped me prepare this work, including, among others my mother, my friends and the authors from which I have drawn various conclusions as well as insight into the fields of neurogenetics as well as neural implants and their possible impact on our society and views of the human life.
  
\end{par}

\newpage

\section{Abstract}

\begin{par}
  
  
  
\end{par}

\newpage


\tableofcontents

\cleardoubleemptypage

\listoffigures
% TODO: Add Tables in case they exist

% TODO: Add ABBR in case they exist

\cleardoubleemptypage

\mainmatter
\chapter{The Idea of Transhumanism}
  \section{Meet the new you}
    \subsection{Reshaping your mind and body}
      \begin{par}
        A sunny summer afternoon, a glass of fresh water with a touch of fresh lemon as well as your favorite weekly magazine. You rest lazily enjoying the day.\\
        What might sound like a scene out of a novel it could be very well a likely situation in the near future. 
        But what makes this so special is that you have just had your 150$^{th}$ birthday. 
        You look as good as ever and don't seem to have aged since your late twenties. 
        This is your prime time it has been going on for some time, and you don't really see an end to it.
      \end{par}
      \\
      \begin{par}
        The example above is one from countless others with which one can charaterize Transhumanism.
        Having a message of well-being deep in it's roots, shows that the Human and their lives are put into the foreground.
        Transhumanism as we know it today is a movement that spans people from the whole world.
        It has its roots in the restlessness of man's willingness to keep pushing the boundaries of what is possible.\cite{bostromhistory}
        As a Transhumanist you believe that it is possible to achieve a higher state of being by \emph{transcending} the status quo. 
        You would still be human, but you transcended yourself as a \emph{Homo Sapiens}.
        In this paper we will explore how Transhumanism believes to solve specific problems in the future.
      \end{par}

    \subsection{The Fields of Transhumanism}
      \begin{par}
       \emph{ In the following pages we will cover various subjects pertaining to the fields cited below. A previous knowledge is not required, but a will to research is still asked.} \\
      \end{par}
      \begin{par}
        Transhumanists believe that as Humans we should use our existing and future technical capabilities to enhance our lives. 
        Thus the fields of \emph{nanotechnology}, \emph{genetics}, \emph{artifical intelligence}, and \emph{computer science} are among the fields of high interest.
      \end{par}
      \begin{par}
        Listing these fields one by one, trying to find a Transhumanist area of activity would be a waste of time. Since the advancement of these various fields has put us in situations where one could not go any further without the others. For example the project of understanding one's brain would require a profound understanding of the molecular level (each neuron), the physical level (a neuron and each of its synapses) and on a behaviour level (the brain as a whole). 
      \end{par}

  \section{A Transhumanist Future}
    \subsection{The Human Body}
      \begin{par}
        In the field of Transhumanism the idea of Bioengineering is a very important Topic. 
        As humans we are limited by our own Body. 
        We only have one and cannot replace it. 
        This means we have to take care of it and try to fix any problems that arise. 
      \end{par}
      \begin{par}
        The past thousands of years if someone lost their legs they would be in a very difficult position. 
        They would have to help their families to survive by being as productive as they ever have been. 
        Obviously this was not always possible and thus that person would have been condemned to a life of misery.
      \end{par}
      \begin{par}
        Today with the help of prosthetics, almost anyone can keep living an almost normal life by wearing artifical legs. 
        This however is still limited though. 
        You wouldn't be able to move them as you would have moved your old legs. 
        You don't feel through them either. 
        This solution is not ideal.
      \end{par}
      \begin{par}
        As a Transhumanist you believe that in the near future Bionics would be advanced enough to practically replace any body part and this not just on a physical level but also on a more deep psychological level.
      \end{par}
      \begin{par}
        This brings us to the next Idea that many Transhumanists hold. \emph{Why just replace?} 
        After all, if we are able to bridge the body-electronics problem, the only limit to our physical bodies would be our own knowledge and imagination. 
        At this point a similar explosion of diversity might occur as it has happened to electronics. 
        That is, many different Limb Modifications could appear on a sort of global market, bringing you features you did not posses beforehand. 
        Even if it sounds like science fiction, but the ability to handle cooking food with your, now bionic, hands might not be far off. 
        And why not cook it directly through your hands? Or get by with solid food all together?
      \end{par}
      \begin{par}
        It is the final advancement, the ability to replace the whole body that would allow for a complete revolution of the human status. 
        If we were able to exit our own bodies and evolve without them we could target our own evolution towards a specific goal.\cite{NBICReport} 
      \end{par}
      \begin{par}
        A more down to earth train of thought would be to think of the ways we could more easily monitor our own bodies. 
        Diabetic patients would be able to lead a normal life with a special pump that is able to take readings much like a normal person's Pancreas would do. 
        Once our understanding is far enough we might even replace the sick Pancreas completely thus curing many health problems that exist throughout the world today.
      \end{par}
      \begin{par}
        But we would not have to stop at our own bodies but also monitor our environments. 
        For example workers that have to work in hazardous environments could have integrated Geiger Counters in their Eyes, allowing them to view disintegrations in real time. 
        An Infrared sensor would also help many professionals as faulty equipment can show signs of overheating thus allowing preemptive reparations before any major accidents.
      \end{par}
    \subsection{The Human Mind}
      \begin{par}
        If someone were to ask you today to multiply two by two, you will obviously give the answer of four. 
        For simple mathematical questions we know the answer almost instantly. Problems usually start once we try to think about higher math problems. 
        What is the answer to $4^4$? The answer is not instantly available to us. 
        The past thousands of years we did not need to solve these types of problems to survive so our brains never had to adapt to it.
      \end{par}
      \begin{par}
        Once we understand how the human brain functions we could draft extensions or replacements of the responsible parts and thus enhance our capabilities of difficult computation. 
        This would be enough in a first moment, but once we realize that it might be possible to completely simulate a human brain we could survive without our bodies.\cite{FutureMinds}
      \end{par}
      \begin{par}
        This realization is core to the Transhumanist Thought since transcending the Human body is one of the goals most aspire to. 
        This has several implications in and of it self as well. 
        Those familiar with Physolophical concepts about the human existence know that there are many currents of thought about the idea of what one might call the mind. 
      \end{par}
      \begin{enumerate}
        \item Materialsm, the idea that a person's being is constituted with the same components its body is. (i.e. Atoms and Molecules)
        \item Spritualism, the idea that a person's being is constituted by a soul that does not live in the body and thus survives after the death of the body.
        \item Continuity, the idea that a person's being is but their past experiences. 
      \end{enumerate}
      \begin{par}
        Now let's imagine that in the near future your mind would be scanned. 
        All of your memories and special connections would be mapped and uploaded to a central computer. 
        Then, taking state of the art transmission this state of yourself would be send to a far away planet. 
        There your mind could then be put into a robot. You would have a new body as the one on earth would have been destroyed. 
        Taking the previous points, in (2. + 3.) you would still be you. After all, your consciousness still exists and only your old body got merely destroyed. 
        After (1.) you would be dead. This consciousness in the robot would be another being. 
        You had left your body behind, thus you have allowed to be killed. 
        This Robot is a different being altogether, which might die as well once it is destroyed.
      \end{par}
  \section{The possible effects}
    \subsection{The Impact of Money}
      \begin{par}

        Explain how rich people might attain a better way of living just because they have more money

      \end{par}
      \begin{par}
        Medication is expensive. It takes decades of research and testing before it gets approved to be used on humans. 
        Along the way it could fail to pass a critical percentage and get rejected.
        With a financial term it could be labeled as \emph{risky}.
        This same process applies to prostethics as well today, and insurance does not always provide full coverage -- leaving the lesser wealthy behind.
      \end{par}
      \begin{par}
        There is no reason to believe that augmentations of the human body would be treated any differently, if anything it might be even worse as most might not be labeled as critical for a good health.
        It is a known theory that more privileged citizens have a tendency to attribute their wealth to their own actions.\cite{MoneyMean} 
        Using augments this could be further entrenched and even marginalize parts the society even more.
        One could imagine a dichotomy between people who have been \emph{augmented} and those who have not solely due to their upbringing.
      \end{par}
  \subsection{The Impact on Crimes}
    \begin{par}
      Committing a crime in the society of today is often going to result into prison.
      But will the punishments of today still be appropriate if our Lifespans triple?
      One could easily imagine someone stealing a particulary large amount of money and receives a jail term of a year.
      If the projected lifespan is about four hundred years, one year is an incredibly small time and might seem misappropriate.
    \end{par}
    \begin{par}
      This stand point is also valid from the other side. If you are sentenced to life and still have two hundred years to life is it still appropriate?
      In today's society the death sentence is a very polemic subject, it might resurface at that point.
    \end{par}
    \subsection{The Impact of Immortality}
      \begin{par}

        Since our ability to survive to new threats is dependant on Evolution we will have to self evolve. This could lead to problems.
  
      \end{par}
      \begin{par}
        Genetic Engineering is being actively today.
        Today the biggest goal of it is therapeutic applications. 
        There are many genetic diseases and we do not understand them all.
        However once we truly understand our genome it is very possible that we, as a human race, can evolve ourselves.
      \end{par}
      \begin{par}
        This self evolution could lead to brains that are more performant, have a higher memory capacity and might be less prone to diseases.
        A higher athletic condition is also possible by activating or substituting some genes at specific conditions.
      \end{par}
  \section{To Resume}
    \begin{par}
      Transhumanism incorporates our current technical advancement and tries to apply it to our own species. 
      Imagining what is possible today and what could be tomorrow.
      This process of trying to make sense of what does not exist yet is fundamental in how we approach science as it allows us to work towards a goal and thus get progress.
      
    \end{par}

\chapter{From the Brain to the Singularity}
  \section{The Superintelligence}
    \subsection{From the Brain to the Computer to the Superbrain}
      \begin{par}
        Moore's law is an often cited observation made in the 60' about the progress of Computer Power. 
        More specifically, Moore stipulated that the number of transistors on integrated circuits doubles approximately every two years. 
        It can be shown that this observation holds true since it has been brought forward.
        This has several implications for future forecasts. 
        One can imagine that through ever progressing technical knowledge a critical amount of calculations per second can be achieved.
        This critical amount would be the equivalent of the human brain.
        We are after all limited through the size and amount of neurons that our brain is made out of.
        This now intelligent Computer could himself design an even more intelligent design or programm of himself.
        Thus, given enough time and energy the intelligence evolves into a superintelligence.
        This process is called the technological singularity due to the unknown factor of what will happen to humankind.
      \end{par}
      \begin{par}
        However there are several hurdles to what will make this intelligence work.
      \end{par}
      \subsubsection{The Program}
        A human, or several, will have to write the program. 
        Which means that it will inherently describe features of humankind.
        This can already be noted today with what is called \emph{Biomimicry}: Humans look at nature and try to replicate it.
        However, many problems arise with this. 
        For one, as Humans we have flaws and these flaws could be very well exacerbated by applying it to a higher machine.
        The problem is also that the same way Chimps cannot understand what we say or do, a superintelligent Computer might be useless to us.
      \subsubsection{The Infrastructure}
        Power today is an important resource, and with the rise of ever faster Computers, this demand for power will rise even more.
        The fastest computer as of today, the Tianhe-2 uses about 20 MW of power per hour. 
        When scaled further up, there will have to be progress for it to be sustainable.
  \section{Harnessing the Torrent of Information}
    \subsection{Possible ways to Control the Singularity}
      \begin{par}
        Controlling something intelligenter than us seems to be a logical contradiction, after all if a chimp can't control us how could we control a superintelligence?
        However to answer this we do not have to go to the levels of a superintelligence, instead one could look how we manage with fellow humans.
        The idea of control becomes obselete once a somewhat complete form intelligence exists.
        As is true with many things, intelligence does have a gradual step to it.
        A fish is able to learn from being attacked by a certain animal that it will die if that happens again.
        However, while we use the word \emph{die} as a way to get our point across it is not true that the fish truly knows what death is.
        Humans are the only species able to clearly recognize that we will die at some point.
        We are thus making preparations almost the whole of our live.
        One cannot thus compare a chimp and a human and thus deduce what will happen between a human and a superintelligence.
        It is however clear that some aspects might escape us.
        The same way that a man who lived two hundred years ago did not have the necessary thought processes to imagine something fictional as true to draw conclusions, our current generation may have trouble understanding a superintelligence.
        There is also no doubt that if we can, at some point understand this superintelligence, then we will.\cite{FutureMinds}
      \end{par}



\cleardoubleemptypage

\backmatter

\bibliography{bibliography}{}
\bibliographystyle{plain}

% TODO: Add Thanks

\end{document}
